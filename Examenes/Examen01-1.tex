\documentclass[10pt,respuestas,a4]{aleph-examen}
% \documentclass[10pt,a4]{aleph-examen}

% -- Paquetes extra
\usepackage{aleph-comandos}
\usepackage{tkz-tab}
\usepackage{pstricks,pst-plot,pst-bezier}
        \psset{showorigin=false}
\usepackage{multicol}

% -- Datos del examen
\universidad{Escuela de Ciencias Físicas y Matemática}
\autor{Mat. Andrés Merino}
\materia{Cálculo diferencia e integral}
\examen{Examen no. 01}
\fecha{Semestre 2020-1}

\logouno[3cm]{Logos/logoPUCE_01_c}
\definecolor{colortext}{HTML}{008DA3}



\begin{document}

\encabezado

\begin{preguntas}
%%%%%%%%%%%%%%%%%%%%%%%%%%%%%%%%%%%%%%%%%%
%%%%%%%%%%%%%%%%%%%%%%%%%%%%%%%%%%%%%%%%%%
%%%%%%%%%%%%%%%%%%%%%%%%%%%%%%%%%%%%%%%%%%
\item
    Resolver las siguientes ecuaciones:\puntaje{2.0}
    \begin{multicols}{2}
        \begin{enumerate}[label=\textit{\alph*)}]
            \item
                $ 2000 = 100 \cdot 3^{6x}$
            \item
                $x+\sqrt{8-2x}=2$
        \end{enumerate}    
    \end{multicols}

\begin{respuesta}
    \begin{enumerate}[label=\textit{\alph*)},leftmargin=*]
    \item
        Tenemos que
        \begin{align*}
            2000 = 100 \cdot 3^{6x}
            & \qDimp \frac{2000}{100} = \cdot 3^{6x}\\
            & \qDimp 20 =  3^{6x}\\
            & \qDimp \ln(20) = \ln(3^{6x})\\
            & \qDimp \ln(20) = 6x\cdot\ln(3)\\
            & \qDimp x = \frac{\ln(20)}{6\ln(3)} \approx 0.45447.
        \end{align*}
    \item
        Tenemos que
        \begin{align*}
            x + \sqrt{8 - 2x} = 2
            & \qDimp \sqrt{8 - 2x} = 2 - x\\
            & \,\qImp (\sqrt{8 - 2x})^2 = (2 - x)^2\\
            & \qDimp 8 - 2x = 4 - 4x + x^2\\
            & \qDimp x^2 - 2x - 4 = 0\\
            & \qDimp x = 1-\sqrt{5} \qlor x = 1+\sqrt{5}\qedhere
        \end{align*}
        Ahora, reemplazando en la ecuación original, podemos comprobar que la única solución es $x = 1-\sqrt{5}$.
    \end{enumerate}
\end{respuesta}
%%%%%%%%%%%%%%%%%%%%%%%%%%%%%%%%%%%%%%%%%%
%%%%%%%%%%%%%%%%%%%%%%%%%%%%%%%%%%%%%%%%%%
%%%%%%%%%%%%%%%%%%%%%%%%%%%%%%%%%%%%%%%%%%
\item
    Un saldo compensatorio se refiere a la práctica en la cual un banco requiere a quien solicita un crédito, mantenga en depósito una cierta parte del préstamo durante el plazo del mismo. Por ejemplo, si una empresa obtiene un préstamo de \$$100\,000$, el cual requiere de un saldo compensatorio del $20$\%, tendría que dejar \$$20\,000$ en depósito y usar sólo \$$80\,000$. Para pago de nómina, una empresa requiere \$$230\,000$ y decide solicitar un préstamo para cubrir esta cantidad, la entidad que dará el préstamo solicita un saldo compensatorio del $15$\%. ¿Cuál debe ser el monto que la empresa debe solicitar para poder cubrir el pago de la nómina?\puntaje{3.0}
    
\begin{respuesta}\hspace{0pt}
    \paragraph{Variables:} Tomemos:
    \begin{itemize}
        \item $m$: monto que la empresa debe solicitar.
    \end{itemize}
    \paragraph{Planteamiento:} La cantidad que la empresa va a recibir es
    \[
        (1 - 0.15)m,
    \]
    para cubrir el pago de nómina, se necesita que
    \[
        (1 - 0.15)m = 230000.
    \]
        
    \paragraph{Resolución:} Resolvemos la ecuación. Despejando tenemos que
    \begin{align*}
       (1 - 0.15)m = 230000
       &\qDimp 0.85m = 230000 \\
       &\qDimp m = \frac{230000}{0.85} \approx 270588.23
    \end{align*}
    
    \paragraph{Respuesta:} Deben solicitarse un monto de \$$270\,588.23$, aproximadamente.
\end{respuesta}

%%%%%%%%%%%%%%%%%%%%%%%%%%%%%%%%%%%%%%%%%%
%%%%%%%%%%%%%%%%%%%%%%%%%%%%%%%%%%%%%%%%%%
%%%%%%%%%%%%%%%%%%%%%%%%%%%%%%%%%%%%%%%%%%
\item
    Resolver la siguiente inecuación:\puntaje{2.0}
    \[
        \frac{3}{6-2x}\leq \frac{2x}{2+x}.
    \]

\begin{respuesta}
    Para resolver la inecuación, primero debemos dejar todos los términos en un solo lado de la inecuación y luego separarla en factores:
    \begin{align*}
       \frac{3}{6-2x}\leq \frac{2x}{2+x}
       &\qDimp \frac{3}{6-2x} - \frac{2x}{2+x} \leq 0 \\
       &\qDimp \frac{3(2+x)-2x(6-2x)}{(6-2x)(2+x)} \leq 0\\
       &\qDimp \frac{4 x^2 - 9 x + 6}{(6-2x)(2+x)} \leq 0.
    \end{align*}
    Con esto, realizamos la tabla de factores:
    \begin{center}\small
    \begin{tabular}{c|c|c|c|c|c}
        & $\l]-\infty,-2\r[$ & $-2$ & $\l]-2,3\r[$ & $3$ &$\l]3,+\infty\r[$\\
        \hline
        $4 x^2 - 9 x + 6$ & $+$ & $+$ & $+$ & $+$ & $+$\\
        $2 +x$ & $-$ & $0$ & $+$ & $+$ & $+$\\
        $6-2x$ & $+$ & $+$ & $+$ & $0$ & $-$\\
        \hline
        & $-$ & $\nexists$ & $+$  & $\nexists$ & -
    \end{tabular}
    \end{center}
    Como buscamos los menores o iguales que $0$, tenemos que la solución es
    \[
        S = \l]-\infty,-2\r[ \cup \l]3,+\infty \r[.
        \qedhere
    \]
\end{respuesta}

%%%%%%%%%%%%%%%%%%%%%%%%%%%%%%%%%%%%%%%%%%
%%%%%%%%%%%%%%%%%%%%%%%%%%%%%%%%%%%%%%%%%%
%%%%%%%%%%%%%%%%%%%%%%%%%%%%%%%%%%%%%%%%%%
\item
    Considere las funciones
    \[
        \funcion{f}{\R}{\R}{x}{2 - 3x - x^2}
        \texty
        \funcion{g}{\R}{\R}{x}{\frac{2x}{x + 2},}
    \]
    determinar, para $x,h\in\R$ y $h\neq 0$,
    \begin{multicols}{2}
        \begin{enumerate}[label=\textit{\alph*)}]
        \item 
            $f(3)$\puntaje{0.3}
        \item
            $g(-2)$\puntaje{0.3}
        \item
            $f(1-x^2)$\puntaje{0.5}
        \item
            $(f\circ g)(x)$\puntaje{0.7}
        \item
            $(g\circ f)(x)$\puntaje{0.7}
        \item
            $\dfrac{f(x+h)-f(x)}{h}$\puntaje{1.0}
        \item
            $\dfrac{g(x+h)-g(x)}{h}$\puntaje{1.0}
        \end{enumerate}    
    \end{multicols}
    
\begin{respuesta}
    \hspace{0pt}
    \begin{enumerate}[label=\textit{\alph*)}]
    \item 
        $f(3) = 2 - 3(3) - (3)^2 = -16$
    \item
        $g(-2)$ no existe
    \item
        $f(1-x^2) = 2 - 3(1-x^2) - (1-x^2)^2 = -x^4 + 5 x^2 - 2$
    \item
        $\begin{aligned}[t]
        (f\circ g)(x) 
        & = f(g(x))\\
        & = f\l(\frac{2x}{x + 2}\r)\\
        & = 2 - 3\l(\frac{2x}{x + 2}\r) - \l(\frac{2x}{x + 2}\r)^2\\    
        & = \frac{-8 x^2 - 4x + 8}{(x + 2)^2}
        \end{aligned}$
    \item
        $\begin{aligned}[t]
        (g\circ f)(x) 
        & = g(f(x))\\
        & = g(2 - 3x - x^2)\\
        & = \frac{2(2 - 3x - x^2)}{(2 - 3x - x^2) + 2}\\
        & = \frac{4 - 6x - 2x^2)}{4 - 3x - x^2}
        \end{aligned}$
    \item
        $\begin{aligned}[t]
        \dfrac{f(x+h)-f(x)}{h}
        & = \dfrac{2 - 3(x+h) - (x+h)^2-(2 - 3x - x^2)}{h}\\
        & = \dfrac{-h^2 - 2 xh - 3h}{h}\\
        & = -h - 2 x - 3
        \end{aligned}$
    \item
        $\begin{aligned}[t]
        \dfrac{g(x+h)-g(x)}{h}
        & = \dfrac{\frac{2(x+h)}{(x+h) + 2}- \frac{2x}{x + 2}}{h}\\
        & = \dfrac{\dfrac{2(x+h)(x + 2)-2x(x+h + 2)}{(x+h + 2)(x + 2)}}{h}\\
        & = \dfrac{\dfrac{4h}{(x+h + 2)(x + 2)}}{h}\\
        & = \dfrac{4}{(x+h + 2)(x + 2)}
        \end{aligned}$
    \end{enumerate}    \qedhere
\end{respuesta}

%%%%%%%%%%%%%%%%%%%%%%%%%%%%%%%%%%%%%%%%%%
%%%%%%%%%%%%%%%%%%%%%%%%%%%%%%%%%%%%%%%%%%
%%%%%%%%%%%%%%%%%%%%%%%%%%%%%%%%%%%%%%%%%%
\item
    Considere la función $\func{f}{[-5,5]}{\R}$ cuya gráfica se muestra a continuación.
    \begin{center}
    \psset{algebraic,plotpoints=500}
    \begin{pspicture}(-5.5,-3.5)(5.5,3.5)
        \psgrid[gridcolor=gray,subgriddiv=0,griddots=5,gridlabels=0pt](-5,-3)(5,3)
        \psaxes[labelFontSize=\scriptstyle,Dx=1,Dy=1]{->}(0,0)(-5.5,-3.5)(5.5,3.5)
        
        \psset{linewidth=1.5pt}
        \psbcurve(-5,-1)(-4,-1.5)(-3,-3)
        \psbcurve(-3,-3)(-2,-1.5)(-1,1)
        \psline(-1,-2)(2,1)
        \psline(2,1)(3,-1)
        \psline(3,3)(5,2)
        
        \psdot*[dotsize=6pt](-1,1)
        \psdot[dotsize=6pt,dotstyle=o](-1,-2)
        
        \psdot*[dotsize=6pt](3,3)
        \psdot[dotsize=6pt,dotstyle=o](3,-1)
    \end{pspicture}
    \end{center}
    \begin{enumerate}[label=\textit{\alph*)}]
    \item
        Determine los intervalos donde es estrictamente creciente.\puntaje{1.0}
    \item
        Determine los intervalos donde es estrictamente decreciente.\puntaje{1.0}
    \item
        Determine el valor de $f(-1)$, $f(0)$ y $f(1)$.\puntaje{0.5}
    \item
        Determine $\img(f)$.\puntaje{0.5}
    \end{enumerate}

\begin{respuesta}
    \hspace{0pt}
    \begin{enumerate}[label=\textit{\alph*)}]
    \item
        La función es estrictamente creciente en $[-3,-1]$ y $\l]-1,2\r]$.
    \item
        La función es estrictamente decreciente en $[-5,-3]$, $\l[2,3\r[$ y $\l[3,5\r]$.
    \item
        Tenemos que $f(-1) = 1$, $f(0) = -1$ y $f(1) = 0$.
    \item
        Tenemos que $\img(f) = [-3,1] \cup [2,3]$.\qedhere
    \end{enumerate}
\end{respuesta}
    
%%%%%%%%%%%%%%%%%%%%%%%%%%%%%%%%%%%%%%%%%%
%%%%%%%%%%%%%%%%%%%%%%%%%%%%%%%%%%%%%%%%%%
%%%%%%%%%%%%%%%%%%%%%%%%%%%%%%%%%%%%%%%%%%
\item
    Dibuje una función $\func{f}{[-3,3]}{\R}$ tal que
    \begin{itemize}
    \item 
        $f(-2)=3$, $f(-1)=-2$, $f(0)=2$, $f(3)=1$;\puntaje{0.8}
    \item 
        sea estricta creciente en $\l[-3,-2\r]$;\puntaje{0.4}
    \item 
        sea estricta decreciente en $\l[-2,-1\r[$;\puntaje{0.4}
    \item 
        sea estricta decreciente en $[-1,3]$.\puntaje{0.4}
    \end{itemize}
    
\begin{respuesta}
    Una posible respuesta es:
    \begin{center}
    \psset{algebraic,plotpoints=500}
    \begin{pspicture}(-3.5,-3)(3.5,3.5)
        \psgrid[gridcolor=gray,subgriddiv=0,griddots=5,gridlabels=0pt](-3,-3)(3,3)
        \psaxes[labelFontSize=\scriptstyle,Dx=1,Dy=1]{->}(0,0)(-3.5,-3.5)(3.5,3.5)
        
        \psset{linewidth=1.5pt}
        \psline(-3,1)(-2,3)
        \psline(-2,3)(-1,-2)
        \psbcurve(-1,3)(0,2)(3,1)
        
        \psdots*[dotsize=6pt](-2,3)(-1,-2)(0,2)(3,1)
        \psdot[dotsize=6pt,dotstyle=o](-1,3)
    \end{pspicture}
    \end{center}
\end{respuesta}


%%%%%%%%%%%%%%%%%%%%%%%%%%%%%%%%%%%%%%%%%%
%%%%%%%%%%%%%%%%%%%%%%%%%%%%%%%%%%%%%%%%%%
%%%%%%%%%%%%%%%%%%%%%%%%%%%%%%%%%%%%%%%%%%
\item
    Una empresa fabrica dos productos: zapatos y zapatillas. Por la infraestructura, la empresa solo puede fabricar $100$ productos en total. Por otro lado, la empresa tiene un gasto fijo semanal de \$$1600$, el costo de fabricar los zapatos es de \$$40$ el par y el de las zapatillas, \$$50$. Cada par de zapatos se vende en \$$60$ y el de zapatillas, \$$75$.
    \begin{enumerate}[label=\textit{\alph*)}]
    \item 
        Modele la utilidad que obtiene la empresa en función de la cantidad de zapatos producidos.\puntaje{2.0}
    \item
        Si se producen $40$ zapatos, ¿cuál es la utilidad de la empresa?\puntaje{0.5}
    \item
        Si se desea tener una utilidad semanal de \$$300$, ¿cuántos pares de zapatos y zapatillas deben producirse?\puntaje{1.0}
    \end{enumerate}
    
\begin{respuesta}\hspace{0pt}
    \begin{enumerate}[label=\textit{\alph*)},listparindent= \parindent, leftmargin=*]
    \item 
        Para el modelamiento, consideremos lo siguiente:
        \paragraph{Variables:}
        \begin{itemize}
        \item
            $x$: cantidad de zapatos que la empresa produce.
        \item
            $y$: cantidad de zapatillas que la empresa produce.
        \item
            $U(x)$: utilidad que obtiene la empresa en función de la cantidad de zapatos producidos.
        \end{itemize}
        \paragraph{Planteamiento:} Tenemos que el ingreso por zapatos y zapatillas es:
        \[
            60x + 75y,
        \]
        respectivamente; por otro lado, los gastos están dados por
        \[
            1600 + 40x + 50y,
        \]
        por lo tanto, la utilidad está dada por (recordando que solo se podrían fabricar un máximo de 100 zapatos):
        \[
            \funcion{U}{\l[0,100\r]}{\R}{x}{60x + 75y -(1600 + 40x + 50y)= 20 x + 25 y - 1600 .}
        \]
        Ahora, dado que solo se pueden producir 100 artículos, es necesario que
        \[
            x+y=100,
        \]
        de donde
        \[
            y = 100 - x,
        \]
        con esto, la función $U$ nos queda
        \[
            \funcion{U}{[0,100]}{\R}
            {x}{20 x + 25 (100 - x) - 1600 = 900 - 5x.}
        \]
    \item
        Evaluemos la función en $x = 40$:
        \[
            U(40) = 900 - 5(40)
            = 700.
        \]
        Por lo tanto, la utilidad de la empresa si se producen $40$ zapato es de \$$700$.
    \item
        Buscamos $x$ tal que $U(x) = 300$, por lo tanto:
        \begin{align*}
        U(x) = 300 
           &\qDimp 900 - 5x = 300 \\
           &\qDimp 5x = -600 \\
           &\qDimp x = -120.
        \end{align*}
        Pero $x$ no está en el dominio de la función, por lo tanto, no existe una cantidad de zapatos y zapatillas que se pueda producir que generen una utilidad de \$$300$.\qedhere
    \end{enumerate}
\end{respuesta}
    

\end{preguntas}
\final


\end{document}
\documentclass[10pt,respuestas,a4]{aleph-examen}
% \documentclass[10pt,a4]{aleph-examen}

% -- Paquetes adicionales
\usepackage{multicol}
\usepackage{aleph-comandos}

% -- Datos 
\institucion{Proyecto Alephsub0}
% \carrera{Ciencia de Datos}
\asignatura{Funciones}
\tema{Resumen 1}
\autor[A. Merino]{Andrés Merino}
\fecha{Semestre 2024-1}

\logouno[0.3\textwidth]{Logos/LogoAlephsub0-02.png}
\definecolor{colortext}{HTML}{0030A1}
\definecolor{colordef}{HTML}{0030A1}
\fuente{montserrat}


% -- Comandos adicionales


\begin{document}

\encabezado

\begin{preguntas}
%%%%%%%%%%%%%%%%%%%%%%%%%%%%%%%%%%%%%%%%%%
%%%%%%%%%%%%%%%%%%%%%%%%%%%%%%%%%%%%%%%%%%
%%%%%%%%%%%%%%%%%%%%%%%%%%%%%%%%%%%%%%%%%%
\item
    Resolver las siguientes ecuaciones:\puntaje{2.0}
    \begin{multicols}{2}
        \begin{enumerate}[label=\textit{\alph*)}]
            \item
                $ 2000 = 100 \cdot 3^{6x}$
            \item
                $x+\sqrt{8-2x}=2$
        \end{enumerate}    
    \end{multicols}

\begin{respuesta}
    \begin{enumerate}[label=\textit{\alph*)},leftmargin=*]
    \item
        Tenemos que
        \begin{align*}
            2000 = 100 \cdot 3^{6x}
            & \qDimp \frac{2000}{100} = \cdot 3^{6x}\\
            & \qDimp 20 =  3^{6x}\\
            & \qDimp \ln(20) = \ln(3^{6x})\\
            & \qDimp \ln(20) = 6x\cdot\ln(3)\\
            & \qDimp x = \frac{\ln(20)}{6\ln(3)} \approx 0.45447.
        \end{align*}
    \item
        Tenemos que
        \begin{align*}
            x + \sqrt{8 - 2x} = 2
            & \qDimp \sqrt{8 - 2x} = 2 - x\\
            & \,\qImp (\sqrt{8 - 2x})^2 = (2 - x)^2\\
            & \qDimp 8 - 2x = 4 - 4x + x^2\\
            & \qDimp x^2 - 2x - 4 = 0\\
            & \qDimp x = 1-\sqrt{5} \qlor x = 1+\sqrt{5}
        \end{align*}
        Ahora, reemplazando en la ecuación original, podemos comprobar que la única solución es $x = 1-\sqrt{5}$.\qedhere
    \end{enumerate}
\end{respuesta}
%%%%%%%%%%%%%%%%%%%%%%%%%%%%%%%%%%%%%%%%%%
%%%%%%%%%%%%%%%%%%%%%%%%%%%%%%%%%%%%%%%%%%
%%%%%%%%%%%%%%%%%%%%%%%%%%%%%%%%%%%%%%%%%%
\item
    Un saldo compensatorio se refiere a la práctica en la cual un banco requiere a quien solicita un crédito, mantenga en depósito una cierta parte del préstamo durante el plazo del mismo. Por ejemplo, si una empresa obtiene un préstamo de \$$100\,000$, el cual requiere de un saldo compensatorio del $20$\%, tendría que dejar \$$20\,000$ en depósito y usar sólo \$$80\,000$. Para pago de nómina, una empresa requiere \$$230\,000$ y decide solicitar un préstamo para cubrir esta cantidad, la entidad que dará el préstamo solicita un saldo compensatorio del $15$\%. ¿Cuál debe ser el monto que la empresa debe solicitar para poder cubrir el pago de la nómina?\puntaje{3.0}
    
\begin{respuesta}\hspace{0pt}
    \paragraph{Variables:} Tomemos:
    \begin{itemize}
        \item $m$: monto que la empresa debe solicitar.
    \end{itemize}
    \paragraph{Planteamiento:} La cantidad que la empresa va a recibir es
    \[
        (1 - 0.15)m,
    \]
    para cubrir el pago de nómina, se necesita que
    \[
        (1 - 0.15)m = 230000.
    \]
        
    \paragraph{Resolución:} Resolvemos la ecuación. Despejando tenemos que
    \begin{align*}
       (1 - 0.15)m = 230000
       &\qDimp 0.85m = 230000 \\
       &\qDimp m = \frac{230000}{0.85} \approx 270588.23
    \end{align*}
    
    \paragraph{Respuesta:} Deben solicitarse un monto de \$$270\,588.23$, aproximadamente.
\end{respuesta}

\end{preguntas}



\end{document}
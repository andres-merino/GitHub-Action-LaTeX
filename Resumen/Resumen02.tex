\documentclass[a4,11pt]{aleph-notas}

% -- Paquetes adicionales
\usepackage{enumitem}
\usepackage{aleph-comandos}

% -- Datos 
\institucion{Proyecto Alephsub0}
% \carrera{Ciencia de Datos}
\asignatura{Funciones}
\tema{Resumen 2}
\autor[A. Merino]{Andrés Merino}
\fecha{Prueba 2024/06/30 - 18:09}

\logouno[0.3\textwidth]{Logos/LogoAlephsub0-02.png}
\definecolor{colortext}{HTML}{0030A1}
\definecolor{colordef}{HTML}{0030A1}
\fuente{montserrat}


% -- Comandos adicionales


\begin{document}

\encabezado

%%%%%%%%%%%%%%%%%%%%%%%%%%%%%%%%%%%%%%%%
\section{Funciones}
%%%%%%%%%%%%%%%%%%%%%%%%%%%%%%%%%%%%%%%%

\begin{defi}[Función]
    Dados $A$ y $B$ dos conjuntos, $f$ es una \textbf{función de $A$ en $B$} si:
    \begin{itemize}
    \item 
        $f\subseteq A\times B$;
    \item 
        para todo $x\in A$, existe $y\in B$ tal que $(x,y)\in f$; y
    \item 
        si $(x,y)\in f$ y $(x,z)\in f$, entonces $y=z$.
    \end{itemize}
    Si $f$ es una función de $A$ en $B$, escribirá $\func{f}{A}{B}$. Y, en lugar de $(x,y)\in f$, escribiremos $f(x)=y$, ya que dado $x$, $y$ es único.
\end{defi}

En otras palabras, $f$ es una función de $A$ en $B$ si es una relación entre los elementos de $A$ y $B$ de modo que para cada elemento $x$ de $A$, hay un único elemento $y$ de $B$ que le corresponde a $x$ en esta relación; a ese elemento $y$ se le llama \textbf{imagen de $x$ respecto de $f$ y se le representa por $f(x)$}.

\begin{defi}[Dominio]
    Dada $\func{f}{A}{B}$ el conjunto $A$ se llama \textbf{dominio} de $f$ y se le representa por $\dom(f)$.
\end{defi}

\begin{defi}[Imagen o recorrido]
    Dada una función $\func{f}{A}{B}$, la \textbf{imagen} o el \textbf{recorrido} de $f$ es el conjunto
    \[
        \{f(x) : x\in A\},
    \]
    que se lo representa por 
    \[
        \img(f)
        \texto
        \rec(f).
    \]
\end{defi}



\end{document}